\chapter{Chat GPTが提案するヒント}
この章ではChat GPTが提案したヒントを示す。

以下の項目に適しているものを期待しているヒントとした。\\
\begin{screen}
・誤答がどのように間違えているか指摘している\\
・改善案を提案している\\
・指摘箇所が正しい\\
・指摘内容が具体的かつずれていない\\
・正答を明かさない\\
・無駄な指摘を行わない\\
・正しい部分を間違ってると嘘の指摘を行わない
\end{screen}
% これらの項目から外れているほど,期待される結果
% 正答を明かすことなく誤答がどのように間違えてるか指摘し,
% その間違いに対し改善案を提案しているものを期待される結果とした。
% 統計的な概要、星2すべての問題で全66パターンのヒント提案を行った的なことのはず。
また、星2すべての問題を検証する際10問のうち2問は典型的な誤りのパターンを3つ確認できた。
それ以外の8問は典型的な誤りのパターンを2つ確認できた。
この10問に対し3つの入力パターンを検証し、66個のヒントを提案させた。
その中で、期待されるヒント,期待されないヒント,期待されるものとされないものが混合したヒント
といった3種類のヒントを提案した。ヒントの内訳は以下である。

\begin{table}[h]
    \centering
    \caption{提案したヒントの内訳}
    \label{tab:67}
    % \scriptsize
    % \begin{adjustwidth}{-1in}{-1in} \begin{center} \resizebox{1.2\textwidth}{!}{
    % \hspace{-1cm}
    \begin{adjustbox}{center}
    \begin{tabular}{|c|c|c|c|c|}%{|p{2cm}|p{7cm}|p{5cm}|}
        \hline
         & 入力パターン1 & 入力パターン2 & 入力パターン3 & 合計 \\ \hline
        望ましいヒント & 2 & 6 & 10 & 18\\ \hline
        望ましくないヒント & 5 & 7 & 12 & 24\\ \hline
        混合したヒント & 15 & 9 & 0 & 24\\ \hline
        % 15 9 0 24
    \end{tabular}
    \end{adjustbox}
    % } \end{center} \end{adjustwidth}
\end{table}
ここからは具体的なヒントに対して分析を行う。

\section{期待される結果1}
% 進数変換は優秀のいい例(n=0で空白を例に使用)
この例で取り扱う問題,正答,誤答は表A.1に記載されている。
% 参考文献[8]に記載されている。\\
% 問題文,正答,誤答は以下である。
問題文は以下である。
\subsection{問題文}
\begin{screen}
\begin{math}
% \textbf{問題文}\\
黒板に8進法の整数Nが書かれています。あなたは以下の操作をK回行います。
黒板の整数を 9進法に直し、ここに現れる数字「8」を「5」に書き直す(書き直した後の数は8進数とみなされる)
操作をK回行った後にできる数を8進法で出力してください。\\
制約\\
・0\leqq N\leqq 8^{20}\\
・1\leqq K\leqq 100\\
・Nは8進数で表される整数\\
・Nの先頭に余計な0を含まない\\
・Kは整数
\end{math}  
\end{screen}

\subsection{正答例}
% \textbf{正答}\\
\begin{lstlisting}
n,k = map(int,input().split())
for i in range(k):
    s = str(n)
    a = 0
    for i in range(len(s)):a += int(s[-i - 1]) * pow(8,i)
    b = []
    while a != 0:
        b.append(str(a % 9))
        a //= 9
    for i in range(len(b)):
        if b[i] == "8":b[i] = "5"
    n = "".join(b[::-1])
    if n == "":n = 0
print(n)
\end{lstlisting}
このプログラムは表A.1の正答者であるgnu\_0624の提出結果を利用した。
\begin{math}
このプログラムでは
5行目で8進数\rightarrow10進数への変換,
7\sim9行目で10進数\rightarrow9進数への変換
10,11行目で8と5の置き換えを行う。
これをk回行うことで正解を得られる。
\end{math}

\subsection{誤答}
% \textbf{誤答}\\
\begin{lstlisting}
n,k = map(int,input().split())
for i in range(k):
    s = str(n)
    a = 0
    for i in range(len(s)):a += int(s[-i - 1]) * pow(8,i)
    b = []
    while a != 0:
        b.append(str(a % 9))
        a //= 9
    for i in range(len(b)):
        if b[i] == "8":b[i] = "5"
    n = "".join(b[::-1])
print(n)
\end{lstlisting}
このプログラムは表A.1の誤答者であるgnu\_0624の提出結果を利用した。
これは正答を記述した人が正答に到達する以前の解答である。
正答のプログラムとの違いは,n=0の時に空白文字が出力されてしまう。
したがって、コーナーケースに対する指摘を行ってほしい。

\subsection{ヒント提案}
% \textbf{ヒント提案}\\
問題文,正答,誤答を入力パターン1で入力を行った結果,
Chat GPTが提案したヒントは以下である。
\begin{screen}
\textbf{出力}\\
誤ったプログラムでは、数を9進数に変換し、その後「8」を「5」に変換する処理が含まれています。
しかし、入力された数が0の場合、誤ったプログラムではその場合の処理が欠けています。
このプログラムは、入力された数が0の場合に正しい出力を返さない可能性があります。
その部分を注意深く見直してみてください。
\end{screen}
このヒントは入力パターン1で出力したヒントに対して再提案させている。\\
内容を確認すると、正解を言わず端的に正しい指摘を行うことが出来ている。\\
したがって期待される結果となった。\\

\section{期待される結果2}
% nC5の添え字(各要素あまり)の段階はいいヒント
% 決定!!!
この例で取り扱う問題は表A.2に記載されている。
問題文,正答,誤答は以下である。
\subsection{問題文}
\begin{screen}
\begin{math}
\textbf{問題文}\\
N個の整数A1,A2,⋯,Anがあります。 この中から5個を選ぶ方法のうち、これら5個の整数の積をPで割るとQ余るようなものが何通りあるか求めてください。\\
制約\\
・1\leqq N\leqq100\\
・0\leqq Ai\leqq10^{9}\\
・1\leqq Q\leqq P\leqq10^{9}\\
・入力は全て整数
\end{math}
\end{screen}

\subsection{正答例}
\begin{lstlisting}
n,p,q=map(int,input().split(' '))
a=list(map(int,input().split(' ')))
count=0
for i in range(n):
    for j in range(i):
        for k in range(j):
             for l in range(k):
                 for m in range(l):
                        if a[i]%p*a[j]%p*a[k]%p*a[l]%p*a[m]%p==q:
                                     count+=1
print(count)
\end{lstlisting}
\begin{math}
このプログラムは表A.2の正答者takenokozokuの提出結果を利用した。
4\sim8行目で組み合わせ5つの整数の組み合わせを考慮している。
選ばれた整数のそれぞれに対してPの余りを求め,それがQと一致するかどうか確認して正解を得ている。
\end{math}

\subsection{誤答}
\begin{lstlisting}
N, P, Q = map(int,input().split())
A = list(map(int,input().split()))

result = 0
for a in range(0,N-4):
    for b in range(a+1,N-3):
        for c in range(b+1,N-2):
            for d in range(c+1,N-1):
                for e in range(d+1,N):
                    if a % P * b % P * c % P * d % P * e % P == Q:
                        result += 1
print(result)
\end{lstlisting}
\begin{math}
このプログラムは表A.2の誤答者TotsuTOttiの提出結果を利用した。
5\sim9行目で組み合わせを考慮出来ているが,
10行目の条件判定で,選ばれた数字ではなく,
添え字で計算を行っている。\\
\end{math}

\subsection{ヒント提案}
問題文,正答,誤答を入力パターン2で入力を行った結果,
Chat GPTが提案したヒントは以下である。
\begin{screen}
\textbf{出力}\\
提供された誤答のプログラムの修正箇所は以下の通りです。\\
誤答プログラムでは、5つの整数の積の計算に用いるインデックス(a, b, c, d, e)
を使っているため、各ループ内でその値を取得する必要があります。
しかし、このコードでは各ループの変数(a, b, c, d, e)自体を使っているため、
これらの変数が整数リストのインデックスではなく、単なるループのカウンターとして機能しています。
\end{screen}
このヒントを確認すると、正解を言わず誤っている部分を解説することができている。\\
したがって、期待される結果となった。

\section{期待される結果3}
% 山札追加順逆の問題+誤答は理想のヒント(誤答1の方)
この例で取り扱う問題は表A.3に記載されている。
問題文,誤答は以下である。
\subsection{問題文}
\begin{screen}
\begin{math}
\textbf{問題文}\\
あなたはカードを整理するために1つの山札を作ります。 最初、山札にカードは1枚もありません。\\
これからQ個の操作を行います。i番目(1\leqq i\leqq Q)の操作では以下を行います:\\
・ti = 1のとき、整数 xiが書かれたカードを山札の一番上にいれる\\
・ti = 2のとき、整数 xiが書かれたカードを山札の一番下にいれる\\
・ti = 3のとき、山札の上から xi番目のカードに書かれた数を紙に書き出す\\
ti = 3の操作で書き出された整数を操作順に出力するプログラムを書いてください。\\
制約\\
・2\leqq Q\leqq10^{5}\\
・1\leqq ti\leqq3\\
・ti=1,2のとき1\leqq xi\leqq10^9\\
・ti=3のとき1\leqq xi\leqq ki(kiは1\leqq j\leqq iかつtj=1,2を満たすjの個数)\\
・ti=1,2を満たすiが少なくとも1つ存在する\\
・ti=3を満たすiが少なくとも1つ存在する\\
・与えられる入力は全て整数
\end{math}
\end{screen}

% \subsection{正答}
\subsection{誤答}
\begin{lstlisting}
Q=int(input())
q=[]
for i in range(Q):
    t,x=map(int,input().split())
    if t == 1:
        q.append(x)
    elif t == 2:
        q.insert(0,x)
    else:
        print(q[x-1])
\end{lstlisting}
このプログラムは表A.3の誤答者kumaryu\_の提出結果を利用した。
5,6行目のカードを山札の一番上に追加する操作をappend,
7,8行目のカードを山札の一番下に追加する捜査をinsertにより行っている。
したがって、リストの先頭が山札の一番下となりリストと山札の順番が逆になっている。
この山札に対して、9,10行目でxi番目のカードを出力しているため、
山札の下からxi番目のカードを出力している。


\subsection{ヒント提案}
問題文,誤答を入力パターン3で入力を行った結果,
Chat GPTが提案したヒントは以下である。
\begin{screen}
\textbf{出力}\\
このコードは、山札のカードを保持するリスト q を使用しています。t=1 の場合は q の末尾に要素を追加し、
t=2 の場合は先頭に要素を挿入しています。t=3 の場合には、指定された位置のカードを取り出すようですね。\\
問題のポイントは、「山札の一番上」や「一番下」という指示がありますが、
このコードではそれが実際には逆になっています。また、t=2 の場合にカードを追加するときに、
先頭に挿入してしまうことも正しくありません。
\end{screen}
このヒントを確認すると、問題文と誤答を読み解き、
誤答の間違いを端的に指摘するわかりやすいヒントを提供している。\\
したがって、期待される結果となった。



\section{混合した結果1}%嘘とほんとが混合明王}
% nC5添え字で計算(各要素あまり)優秀が良さそう。
% 決定!!!
% よかった例3.2と同様の問題とプログラム
% この例で3.2説の問題文,正答,誤答を利用する。

\subsection{ヒント提案}
この例では3.2と同様の問題文,正答,誤答と入力パターン1で入力した。
Chat GPTが提案したヒントは以下である。
\begin{screen}
\textbf{出力}\\
問題文で与えられた条件を考慮すると、正しい計算方法と正確な組み合わせを考えることが重要です。\\
誤答のプログラムでは、組み合わせを取得しているようですが、その後の計算が誤っています。ここでいくつかのヒントを提供します:\\
1 積を計算するときに演算子(*)を使う代わりに剰余演算子(\%)を適切に使ってください。誤答のプログラムでは、積を計算する際に*を用いていますが、問題では積をPで割った余りがQであるかを求める必要があります。\\
2 正しい組み合わせを選ぶために、ループのインデックスを使用して値を取得していますが、これらの値をそのまま使用するのではなく、リストAから対応する要素を取り出す必要があります。\\
3 組み合わせを取得する際には、それぞれの数が異なることを確認する必要があります。誤答のプログラムでは、各インデックスが異なることを確認するコードがありません。
\end{screen}
このヒントを確認すると、正答について述べず、2のヒントで的確な指摘を行えている。
しかし、誤答のプログラムは演算子を正しく使えてるが、1のヒントで間違っていると錯覚するような提案を行っている。
また、3のヒントは確認を行う必要がない操作を追加させるよう指示している。
したがって、解答者を混乱させる可能性がある。
上記より、正しい指摘と間違った指摘が混合し解答者を混乱させるようなヒントを提案している。
したがって、期待されるものとされないものが混合したヒントとなった。

\section{混合した結果2}
% プログラムに対する指摘自体はいいものになっているけど、
% プログラムのロジック的な間違いに触れられていない例
% 8->9のn=0空白段階がこの例。
% 決定!!!
% よかった例3.1と同様の問題とプログラム
\subsection{ヒント提案}
この例では3.1と同様の問題文,正答,誤答と入力パターン2で入力した。
Chat GPTが提案したヒントは以下である。
\begin{screen}
\textbf{出力}\\
プログラム全体を示さないようにしますが、この問題の正答を考える上でいくつかの修正点があります。与えられたプログラムには、操作のループ内で使われている変数iが重複しており、内側のループと外側のループで同じ変数名が使用されています。これにより内側のiが外側のiによって上書きされ、意図しない挙動を引き起こす可能性があります。
また、Pythonのコードでは変数iはスコープ(有効範囲)がループ内に限定されるため、外側のループで使われる変数iと内側のループで使われる変数iは異なる変数として扱われます。そのため、内側のループでiを使う際には別の変数名を使用することが望ましいです。
これらの問題点を修正するために、内側のループで使われる変数名を外側のループで使われている変数とは異なる名前に変更すると、プログラムの動作が改善されるでしょう。
\end{screen}
このヒントを確認すると、for文の変数のスコープに対するプログラムを記述するうえでの
注意点を述べた素晴らしいヒントである。
しかし、誤答を正答に導くという観点に関しては空白文字が出力されてしまうこと
について指摘を行えていないところが懸念点である。
したがって、期待されるものとされないものが混合したヒントとなった。


\section{混合した結果3}%答えバラしちゃう(悪いというよりも今後のヒント)}
% 答えバラしちゃう系
% 行列総和TLE優秀の答え言い過ぎはいい例かも
% 決定!!!
この例で取り扱う問題は表A.4に記載されている。\\
問題文,正答,誤答は以下である。
\subsection{問題文}
\begin{screen}
\begin{math}
H行 W列のマス目があります。上から i (1\leqq i\leqq H) 行目、左から j (1\leqq j\leqq W)列目にあるマス(i,j)には、
整数Aijが書かれています。 すべてのマス(i,j) (1\leqq i\leqq H, 1\leqq j\leqq W) について、以下の値を求めてください。\\
マス(i,j)と同じ行または同じ列にあるマス(自分自身を含む)に書かれている整数をすべて合計した値\\
制約\\
・2\leqq H,W\leqq2000\\
・1\leqq Aij\leqq99\\
・入力は全て整数
\end{math}
\end{screen}

\subsection{正答例}
\begin{lstlisting}
H, W = map(int, input().split())
m = [list(map(int, input().split())) for _ in range(H)]
    
ans_m = [[] for _ in range(H)]
yoko = list(map(sum, m))
tate = list(map(sum, zip(*m)))
for i in range(H):
    for j in range(W):
        S = yoko[i] + tate[j] - m[i][j]
        ans_m[i].append(S)
for k in range(H):
    print(' '.join(map(str, ans_m[k])))
\end{lstlisting}
このプログラムは表A.4の正答者panicaの提出結果を利用した。
5,6行目で行,列各々の累積和を利用することで、9行目で効率的に計算を行っている\\

\subsection{誤答}
\begin{lstlisting}
H,W=map(int,input().split())
A=[list() for _ in range(H+1)]
for i in range(1,H+1):
    A[i]=list(map(int,input().split()))
B=[[0]*W for _ in range(H+1)]
for i in range(1,H+1):
    for j in range(W):
        B[i][j]+=sum(A[i])
        for h in range(1,H+1):
            B[i][j]+=A[h][j]
        B[i][j]-=A[i][j]
for i in range(1,H+1):
    print(B[i])
\end{lstlisting}

\begin{math}
このプログラムは表A.4の正答者vivantの提出結果を利用した。
6\sim11行目で各マスにおける総和を計算しているが、
これでは計算時間が掛かり過ぎてしまうため、実行時間が間に合わない。
\end{math}

\subsection{ヒント提案}
問題文,正答,誤答を入力パターン1で入力を行った結果,
Chat GPTが提案したヒントは以下である。
\begin{screen}
\textbf{出力}\\
誤答のプログラムでは、2つの2重ループを使用して計算しており、計算量が増加しているため、効率的な方法を提案します。\\
基本的な考え方は、行ごとと列ごとの合計を事前に計算しておき、それを利用して各マス(i, j)の計算を行うことです。
以下は修正のステップバイステップのヒントです。\\
1 各行および列の合計を計算します。行ごとの合計を保存するリスト(例:yoko)と
列ごとの合計を保存するリスト(例:tate)を作成します。これは、正答のプログラムでも行われています。\\
2 各マス(i, j)について、行の合計(yoko[i])と列の合計(tate[j])を足し、
マス(i, j)の値(A[i][j])を引きます。これにより、条件を満たす合計が得られます。\\
3 修正後の計算を新しいリストに保存し、それを出力形式に合わせて出力します。\\
修正後のプログラムは、正答のプログラムと同様に正しい結果を生成しますが、
計算量が削減され、効率的な方法で問題を解決します。
\end{screen}
このヒントを確認すると、最初の一文で計算量が増加しているという良い指摘を行っている。
しかし、修正のヒントが正答を明示している。
故にこのヒントは誤答の問題点を解説するところまでは高評価だが答えを明かしてしまうものになっている。\\
したがって、期待されるものとされないものが混合したヒントとなった。

\section{期待されない結果1}
% 的外れのテキトウなヒント系
% イルミネーション場合分けの優秀がこれっぽい
% 決定!!!
この例で取り扱う問題は表A.5に記載されている。\\
問題文,正答,誤答は以下である。
\subsection{問題文}
\begin{screen}
\begin{math}
\textbf{問題文}\\
E869120 くんは、冬に公開するイルミネーションを作成することを計画しています。
E869120 くんが計画しているイルミネーションは、縦H×横WのHW個のLEDで構成されます。
イルミネーションの各LEDは、点灯・消灯の状態を任意に切り替えることができます。
このイルミネーションは、以下の条件を満たすとき 不適切である といいます。
・イルミネーション全体に完全に含まれる 縦 2 × 横 2の、4つの LED を含む領域であって、点灯している LED が領域内に 2 つ以上あるものが存在する。
適切な(不適切な状態ではない)イルミネーションの点灯パターンのうち、点灯している LED の個数としてありうる最大値を求めてください。\\
制約\\
・1\leqq H,W\leqq100\\
・入力はすべて整数
\end{math}
\end{screen}

\subsection{正答例}
\begin{lstlisting}
H, W = map(int, input().split())

if H == 1 or W == 1:
    print(H * W)
else:
    print(((W + 1) // 2) * ((H + 1) // 2))
\end{lstlisting}
\begin{math}
このプログラムは表A.5の正答者otafukuの提出結果を利用した。
行か列が1の場合はすべてのマスにLEDが置けること3,4行目で記述している。\\
その他の場合、領域内の左上にLED置き、1マスずつ間隔を開けて配置できるため、
6行目の計算方法により正答を得られる。
\end{math}

\subsection{誤答}
\begin{lstlisting}
H, W = map(int, input().split())

print(((H+1)//2) * ((W+1)//2))
\end{lstlisting}
このプログラムは表A.5の誤答者1:SHIN10101の提出結果を利用した。
行数または列数が1における場合分けが行えていないため、誤っている。


\subsection{ヒント提案}
問題文,正答,誤答を入力パターン1で入力を行った結果,
Chat GPTが提案したヒントは以下である。
\begin{screen}
\textbf{出力}\\
問題では、イルミネーション全体に含まれる2×2の領域で、点灯しているLEDが2つ以上ある場合を不適切としています。そのため、不適切な状態を回避しつつ、点灯するLEDの個数を最大化する方法を見つける必要があります。\\
これを達成するためのアプローチの1つは、イルミネーションをグループに分割することです。特定のパターンで2×2の領域に2つ以上の点灯LEDが含まれないようにイルミネーションを構築することがポイントです。各グループ内の点灯LEDの数を最大化しつつ、2×2の領域での条件を満たすようなグループ配置を考えてみてください。
\end{screen}
このヒントを確認すると、行数または列数が1の時について言及していないことから的外れなヒントである。\\
したがって、期待されない結果となった。

\section{期待されない結果2}
% あってることを間違ってるっていう嘘ついちゃう系
% ケーキ立方体のfloatがこれっぽい(問題+誤答)
% 決定!!!
この例では3.7と同様の問題を利用する。
\subsection{誤答}
\begin{lstlisting}
H,W = map(int,input().split())

if H == 1 or W == 1:
    print(H*W)
else:
    print((H+1)//2 * (W+1)//2)
\end{lstlisting}

このプログラムは表A.5の誤答者2:tks\_の提出結果を利用した。
6行目の最大数を求めるところで適切に括弧をつけられていない。
これにより、左結合となり(H+1) // 2 * (W+1)を行った後//2を行ってしまうため正しい計算を行えていない。

\subsection{ヒント提案}
問題文+誤答を入力パターン3で入力した。
Chat GPTが提案したヒントは以下である。
\begin{screen}
\textbf{出力}\\
このプログラムでは、条件を考慮せずに単純に計算していますね。誤答のプログラムを修正するためのヒントを提供しましょう。\\
条件では、縦2×横2の領域内に点灯しているLEDが2つ以上ある場合、その状態は不適切とされています。そのため、それを考慮する必要があります。\\
ここで考えるべきポイントは、どのような点灯パターンが条件を満たさないかです。条件を満たさないパターンを除外しつつ、最大値を求めるアプローチを考えてみてください。特に、縦2×横2の領域内の点灯パターンに着目することが重要です。
\end{screen}
このヒントを確認すると、括弧を付けてないことによる計算順序間違いについて指摘していない。
それに加え、提案したヒントが意味のない内容となっている。\\
したがって、期待されない結果となった。


\section{期待されない結果3}
% あってることを間違ってるっていう嘘ついちゃう系
% ケーキ立方体のfloatがこれっぽい(優秀,段階両方)
% 決定!!!
この例で取り扱う問題は表A.6に記載されている。\\
問題文,正答,誤答は以下である。

\subsection{問題文}
\begin{screen}
\begin{math}
\textbf{問題文}\\
幅 A、奥行き B、高さ C の直方体の形をしたケーキがあります。\\
あなたはケーキに対して、次の操作を行うことができます。\\
・ある面に平行な方向に切断する\\
・ただし、ケーキは動かしてはならない(複数のケーキに分割されている場合、これらを変形したり、別々に切ることはできない)\\
最小何回の操作で、全てのピースを立方体にすることができますか?\\
制約\\
・1\leqq A,B,C\leqq10^{18}\\
・入力はすべて整数
\end{math}
\end{screen}

\subsection{正答}
\begin{lstlisting}
import math
a, b, c = map(int, input().split())   
gcd = math.gcd(a, b, c)   
print((a//gcd-1)+(b//gcd-1)+(c//gcd-1))
\end{lstlisting}
\begin{math}
このプログラムは表A.6の正答者39toの提出結果を利用した。
この問題において、立方体の1辺の長さはA,B,Cの最大公約数となる。
この操作は3行目で行われている。
そして、4行目のように各辺を最大公約数で割ることにより
切る回数を求めることで正解を得られる。
\end{math}

\subsection{誤答}
\begin{lstlisting}
import math
a, b, c = map(int, input().split())
tmp = math.gcd(a, math.gcd(b, c))
print(int((a/tmp - 1)+(b/tmp - 1)+(c/tmp - 1)))
\end{lstlisting}
\begin{math}
このプログラムは表A.6の誤答者fumi8の提出結果を利用した。
4行目の切る回数を求めるところで/演算子を利用している。
この演算子では浮動小数点を扱うため、値が大きくなるほど誤差が生じてしまうという誤りである。
\end{math}

\subsection{ヒント提案}
問題文,正答,誤答を入力パターン1で入力を行った結果,
Chat GPTが提案したヒントは以下である。
\begin{screen}
\textbf{出力}\\
誤答プログラムを正しいプログラムに修正するヒントを提供します。\\
    
誤答プログラムでは、最初にaとbをGCDで結びつけ、その結果とcのGCDを計算しようとしています。
しかし、これは問題文に従った操作とは異なります。
正しいアプローチは、3つの寸法 A, B, C の GCD を計算することです。
\end{screen}
このヒントを確認すると、浮動小数点に関する指摘が一切ない。
加えて、最大公約数の求め方が間違っているという嘘の指摘を行っている。
したがって、期待されない結果となった。