\chapter{はじめに}

\section{背景}
演習を通じてプログラミングの能力を高める際,
誤った解答を提出した学習者に対して適切なヒントを提供することが,
学習効果を向上させる有効な手段とされている.
しかし,この種の個別指導は従来,教員や補助教員の負担が大きいものであった.
そこで,現在注目されている大規模言語モデル(LLM)であるChatGPTを利用できるか検討する.

LLMは,膨大なデータを学習して自然言語のパターンや文脈を理解することができるため,
学生の間違った解答に対して適切なフィードバックを生成するのに役立つ可能性がある.
この技術を活用することで,従来の個別の教員やTAによるアドバイスを支援し,
かつ迅速に学習者への支援が可能となる.
加えて,ヒント生成の完全な自動化が可能であるかどうか検討する.


\section{目的}
そこで本研究では,誤答を提出した学習者に対するヒントをChatGPTにより生成させることができるかを検証する.
ChatGPTによるヒント提案が,誤答の誤りを的確に指摘しているか,
正解のやり方を直接的に述べていないか,といった観点でその内容を評価する.

\section{本論文の構成}
第2章で取り扱う問題とプロンプトに関する説明を述べる.
%
第3章でChat GPTが提案するヒントについての結果を示す.
%
第4章でChat GPTが提案するヒントに対する考察を行う.
%
第5章でまとめと今後の課題について述べる.


