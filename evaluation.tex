\chapter{Chat GPTが提案するヒントに対する考察}
3章でChat GPTが提案するヒントを示した。
期待される結果のパターンを大きく分けると、
\begin{screen}
・誤答を解説し、修正案を提案する\\
・誤答の誤り部分を端的に指摘している
\end{screen}
上記のように理想的なヒントを提案することがわかった。\\
対して、期待されない結果,混合している結果のパターンを大きく分けると、\\
\begin{screen}
・正しい指摘と間違った指摘が混合し解答者を混乱させる\\
・ヒント自体は素晴らしい内容だが、指摘してほしい内容がずれている\\
・誤答の誤り部分を指摘しているが修正案として答えを明かす\\
・修正点を指摘せず中身のない的外れなもの\\
・誤答の正しい操作を間違っていると指摘してしまう最悪なもの
\end{screen}
といった惜しいものや最悪のものが提案された。\\

 3.9で提案したヒントについて、うまくいかなかった理由として問題文からアルゴリズムを推測しにくいことが考えられる。
最小何回の操作で、すべてのピースを立方体にできますか?という問題文を見ただけでは解法が
すぐに思いつくのはなかなか難しい。それに対して、うまくいった例の3パターンすべては
どのようなプログラムを記述すればよいか問題文から比較的推測しやすい。
したがって、問題文の推測しやすさは重要なポイントである可能性が高い。\\
 3.5で提案したヒントについて、内容は素晴らしいものであったが指摘内容はずれていたことから、
Chat GPTがプログラムの可読性などプログラムを書くうえで重要なポイントを重要視している可能性が高いことがわかった。
したがって、ヒント提案させる際はプログラムを記述するうえでの重要ポイントに注意したプログラムを
与えることがより良いヒントを提案させるために重要であると考えられる。\\
 3.6で提案したヒントについて、誤答の問題点を指摘できていることは素晴らしいことである。
しかし、答えを明かし過ぎているため、プロンプトの改善により答えを言わせないようにするのが
今後の課題である。\\
 3.8で提案したヒントについて、問題文と誤答から具体的なプログラム動作を推測しヒント提案させる必要があった。
しかしこれはプログラムの表面的な指摘ではなく具体的な指摘であるため難易度が高いヒント提案と考えられる。\\
 3.9で提案したヒントについて、正答に影響され過ぎた結果誤答の正しい部分に対して誤った指摘を行っていることが考えられる。
したがって、正解を与えることが悪影響になってる可能性が高い。
しかし、3.9は問題文だけでは解法が想像しにくいため難しい点である。
故に、プロンプトの改善が今後の課題となってくる。\\
 入力パターンに関して、今回取り上げた事例では入力パターン1がうまくいかないことが多かった。
よって、段階的に質問を行いヒント提案させることがより良い手段であると考えられる。\\
 総じて、この研究を取り組みかかった段階では悪い指摘もそれなりにしているが,
いい指摘も行えることがわかったため、人間の監修が必要ではあるが
プロンプトを改善することで誰でも無料で学習環境を整えられる可能性が見出された。
考察の結果,ChatGPT に適切に質問することにより
プログラムの誤りを解説・指摘できる事例を発見したため,
本手法の可能性が明らかになった.
% 今は完璧ではないけどこれだけのことはできることが分かった。

% ・いいヒントを与えることもあるが、的外れなヒントを与えることもある。
% ・正答がプログラムを修正するためのヒントにかなり影響している。
% ・入力を変更するとより良いヒントをもらえることが分かった。\\
% ・今回のヒントでは、入力パターン1のヒントが効果的であったため、
% 段階的に質問しなくてもいいヒントを得られることが分かった。\\
% ・答え(正答)を言わせないようにすればもう少しいいヒントが提案できる\\
% ・"正答のプログラムについて言及せず、改めてヒントを提案してください"
%  と会話すれば正答については述べなくなることがわかりました。\\
%  ・いきなりすべてを与えるのではなく、対話したほうがいいヒント出せそうだと思いました。\\
% ・問題次第では問題文と誤答のみでいいヒントを提案できそうなことがわかりました。\\
% ・問題+正答+誤答でうまくヒント提案できない場合、問題+誤答はいいヒントを出す傾向があります。\\
% ・プログラムの書き方が同じであれば、比較する部分が少なくなるため、いいヒントを出しやすくなると思います。\\
% ・答えをばらした後再提案させるとうまく可能性がある。
% ・ヒント提案の自動化は危ないと感じます。
% ・正答と誤答を比較し解説はできるので、ヒントの下書きくらいには使える性能をしていると思います。