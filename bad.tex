\chapter{Chat GPTが提案したヒントにおける問題,ソースコードなど}

本付録では、Chat GPTが提案してヒントおける問題,ソースコードなど必要な情報を記載する。

\begin{table}[h]
    \centering
    \caption{期待される結果1}
    \label{tab:67}
    % \scriptsize
    % \begin{adjustwidth}{-1in}{-1in} \begin{center} \resizebox{1.2\textwidth}{!}{
    % \hspace{-1cm}
    \begin{adjustbox}{center}
    \begin{tabular}{|c|c|p{10cm}|}%{|p{2cm}|p{7cm}|p{5cm}|}
        \hline
        067 & Base 8 to 9 & \url{https://atcoder.jp/contests/typical90/tasks/typical90_bo} \\ \hline
        gnu\_0624 &正答 & \url{https://atcoder.jp/contests/typical90/submissions/47834422} \\ \hline
        % \begin{lstlisting}[numbers=none,frame=]
        % n,k = map(int,input().split())
        % for i in range(k):
        %     s = str(n)
        %     a = 0
        %     for i in range(len(s)):
        %       a += int(s[-i - 1]) * pow(8,i)
        %     b = []
        %     while a != 0:
        %         b.append(str(a % 9))
        %         a //= 9
        %     for i in range(len(b)):
        %         if b[i] == "8":b[i] = "5"
        %     n = "".join(b[::-1])
        %     if n == "":n = 0
        % print(n)
        % \end{lstlisting} 
        % & \url{https://atcoder.jp/contests/typical90/submissions/47834422} \\ \hline
        gnu\_0624 &誤答
        % \begin{lstlisting}[numbers=none,frame=]
        % n,k = map(int,input().split())
        % for i in range(k):
        %     s = str(n)
        %     a = 0
        %     for i in range(len(s)):
        %       a += int(s[-i - 1]) * pow(8,i)
        %     b = []
        %     while a != 0:
        %         b.append(str(a % 9))
        %         a //= 9
        %     for i in range(len(b)):
        %         if b[i] == "8":b[i] = "5"
        %     n = "".join(b[::-1])
        % print(n)
        % \end{lstlisting} 
        & \url{https://atcoder.jp/contests/typical90/submissions/47834406} \\ \hline
    \end{tabular}
    \end{adjustbox}
    % } \end{center} \end{adjustwidth}
\end{table}



\begin{table}[h]
    \centering
    \caption{期待される結果2}
    \label{tab:55}
    % \scriptsize
    % \begin{adjustwidth}{-1in}{-1in} \begin{center} \resizebox{1.2\textwidth}{!}{
    % \hspace{-1cm}
    \begin{adjustbox}{center}
    \begin{tabular}{|c|c|p{10cm}|}%{|p{2cm}|p{7cm}|p{5cm}|}
        \hline
        055 & Select 5 & \url{https://atcoder.jp/contests/typical90/tasks/typical90_bc} \\ \hline
        takenokozoku &正答 
        % \begin{lstlisting}[numbers=none,frame=]
        % n,p,q=map(int,input().split(' '))
        % a=list(map(int,input().split(' ')))
        % count=0
        % for i in range(n):
        %     for j in range(i):
        %         for k in range(j):
        %             for l in range(k):
        %                 for m in range(l):
        %                     if a[i]%p*a[j]%p*a[k]%p*a[l]%p*a[m]%p==q:
        %                         count+=1
        % \end{lstlisting} 
        & \url{https://atcoder.jp/contests/typical90/submissions/47566614} \\ \hline
        TotsuTotti &誤答 
        % \begin{lstlisting}[numbers=none,frame=]
        % N, P, Q = map(int,input().split())
        % A = list(map(int,input().split()))
            
        % result = 0
        % for a in range(0,N-4):
        %     for b in range(a+1,N-3):
        %         for c in range(b+1,N-2):
        %             for d in range(c+1,N-1):
        %                 for e in range(d+1,N):
        %                     if a % P * b % P * c % P * d % P * e % P == Q:
        %                        result += 1
        % print(result)
        % \end{lstlisting} 
        & \url{https://atcoder.jp/contests/typical90/submissions/35262407} \\ \hline
    \end{tabular}
    \end{adjustbox}
    % } \end{center} \end{adjustwidth}
\end{table}

% \bibitem{山札問題}競プロ典型90門, "061 - Deck", \url{https://atcoder.jp/contests/typical90/tasks/typical90_bi}%, \today 閲覧.
% \bibitem{山札誤答}kumaryu\_  "提出 \#35262407”. \url{https://atcoder.jp/contests/typical90/submissions/35262407}

\begin{table}[h]
    \centering
    \caption{期待される結果3}
    \label{tab:61}
    % \scriptsize
    % \begin{adjustwidth}{-1in}{-1in} \begin{center} \resizebox{1.2\textwidth}{!}{
    % \hspace{-1cm}
    \begin{adjustbox}{center}
    \begin{tabular}{|c|c|p{10cm}|}%{|p{2cm}|p{7cm}|p{5cm}|}
        \hline
        061 & Deck & \url{https://atcoder.jp/contests/typical90/tasks/typical90_bi} \\ \hline
        kamaryu\_ &誤答 
        % \begin{lstlisting}[numbers=none,frame=]
        % Q=int(input())
        % q=[]
        % for i in range(Q):
        %     t,x=map(int,input().split())
        %     if t == 1:
        %         q.append(x)
        %     elif t == 2:
        %         q.insert(0,x)
        %     else:
        %         print(q[x-1])
        % \end{lstlisting} 
        & \url{https://atcoder.jp/contests/typical90/submissions/35262407} \\ \hline
    \end{tabular}
    \end{adjustbox}
    % } \end{center} \end{adjustwidth}
\end{table}




\begin{table}[h]
    \centering
    \caption{混合した結果3}
    \label{tab:4}
    % \scriptsize
    % \begin{adjustwidth}{-1in}{-1in} \begin{center} \resizebox{1.2\textwidth}{!}{
    % \hspace{-1cm}
    \begin{adjustbox}{center}
    \begin{tabular}{|c|c|p{10cm}|}%{|p{2cm}|p{7cm}|p{5cm}|}
        \hline
        004 & Cross Sum & \url{https://atcoder.jp/contests/typical90/tasks/typical90_d} \\ \hline
       panica &正答 
        % \begin{lstlisting}[numbers=none,frame=]
        % H, W = map(int, input().split())
        % m = [list(map(int, input().split())) for _ in range(H)]
                
        % ans_m = [[] for _ in range(H)]
        % yoko = list(map(sum, m))
        % tate = list(map(sum, zip(*m)))
        % for i in range(H):
        %     for j in range(W):
        %         S = yoko[i] + tate[j] - m[i][j]
        %         ans_m[i].append(S)
        % for k in range(H):
        %     print(' '.join(map(str, ans_m[k])))
        % \end{lstlisting} 
        & \url{https://atcoder.jp/contests/typical90/submissions/45978954} \\ \hline
        vivant &誤答 
        % \begin{lstlisting}[numbers=none,frame=]
        % H,W=map(int,input().split())
        % A=[list() for _ in range(H+1)]
        % for i in range(1,H+1):
        %     A[i]=list(map(int,input().split()))
        % B=[[0]*W for _ in range(H+1)]
        % for i in range(1,H+1):
        %     for j in range(W):
        %         B[i][j]+=sum(A[i])
        %         for h in range(1,H+1):
        %             B[i][j]+=A[h][j]
        %         B[i][j]-=A[i][j]
        % for i in range(1,H+1):
        %     print(B[i])
        % \end{lstlisting} 
        & \url{https://atcoder.jp/contests/typical90/submissions/45946123} \\ \hline
    \end{tabular}
    \end{adjustbox}
    % } \end{center} \end{adjustwidth}
\end{table}


\begin{table}[h]
    \centering
    \caption{期待されない結果1}
    \label{tab:33}
    % \scriptsize
    % \begin{adjustwidth}{-1in}{-1in} \begin{center} \resizebox{1.2\textwidth}{!}{
    % \hspace{-1cm}
    \begin{adjustbox}{center}
    \begin{tabular}{|c|c|p{10cm}|}%{|p{2cm}|p{7cm}|p{5cm}|}
        \hline
        033 & Not Too Bright & \url{https://atcoder.jp/contests/typical90/tasks/typical90_ag} \\ \hline
        otafuku & 正答
        % \begin{lstlisting}[numbers=none,frame=]
        % H, W = map(int, input().split())
        %   if H == 1 or W == 1:
        %     print(H * W)
        % else:
        %   print(((W + 1) // 2) * ((H + 1) // 2))
        % \end{lstlisting} 
        & \url{https://atcoder.jp/contests/typical90/submissions/47326798} \\ \hline
        SHIN10101 & 誤答1
        % \begin{lstlisting}[numbers=none,frame=]
        % H, W = map(int, input().split())
        
        % print(((H+1)//2) * ((W+1)//2))
        % \end{lstlisting} 
        & \url{https://atcoder.jp/contests/typical90/submissions/47172648} \\ \hline
        tke\_ & 誤答2
        % \begin{lstlisting}[numbers=none,frame=]
        % H,W = map(int,input().split())
        
        % if H == 1 or W == 1:
        %   print(H*W)
        % else:
        %   print((H+1)//2 * (W+1)//2)
        % \end{lstlisting}
        & \url{https://atcoder.jp/contests/typical90/submissions/47125294} \\ \hline
    \end{tabular}
    \end{adjustbox}
    % } \end{center} \end{adjustwidth}
\end{table}


% \bibitem{ケーキ}競プロ典型90問 “022 - Cubic Cake”. \url{https://atcoder.jp/contests/typical90/tasks/typical90_v} 
% \bibitem{ケーキ正答}39to “提出 \#46888797”. \url{・https://atcoder.jp/contests/typical90/submissions/46888797}
% \bibitem{ケーキ誤答}fumi8  "提出 \#45349168”. \url{https://atcoder.jp/contests/typical90/submissions/45349168}     


\begin{table}[h]
    \centering
    \caption{期待していない結果3}
    \label{tab:22}
    % \scriptsize
    % \begin{adjustwidth}{-1in}{-1in} \begin{center} \resizebox{1.2\textwidth}{!}{
    % \hspace{-1cm}
    \begin{adjustbox}{center}
    \begin{tabular}{|c|c|p{10cm}|}%{|p{2cm}|p{7cm}|p{5cm}|}
        \hline
        022 & Cubic Cake & \url{https://atcoder.jp/contests/typical90/tasks/typical90_v} \\ \hline
        39to &正答 &
        % \begin{lstlisting}[numbers=none,frame=]
        % import math
        
        % a, b, c = map(int, input().split())   
        % gcd = math.gcd(a, b, c)   
        % print((a//gcd-1)+(b//gcd-1)+(c//gcd-1))
        % \end{lstlisting} 
        % & 
        \url{https://atcoder.jp/contests/typical90/submissions/46888797} \\ \hline
        fumi8 &誤答 
        % \begin{lstlisting}[numbers=none,frame=]
        % import math
        % a, b, c = map(int, input().split())
        % tmp = math.gcd(a, math.gcd(b, c))
        % print(int((a/tmp - 1)+(b/tmp - 1)+(c/tmp - 1)))
        % \end{lstlisting} 
        & \url{https://atcoder.jp/contests/typical90/submissions/45349168} \\ \hline
    \end{tabular}
    \end{adjustbox}
    % } \end{center} \end{adjustwidth}
\end{table}




% この章ではChat GPTがうまくヒント提案できなかった例をいくつか挙げていく。

% \section{今後うまくいきそうな例1}%嘘とほんとが混合明王}
% % nC5添え字で計算(各要素あまり)優秀が良さそう。
% % 決定!!!
% % よかった例3.2と同様の問題とプログラム
% % この例で3.2説の問題文,正答,誤答を利用する。

% \subsection{ヒント提案}
% この例では3.2と同様の問題文,正答,誤答と入力パターン1で入力した。
% Chat GPTが提案したヒントは以下である。
% \begin{screen}
% \textbf{出力}\\
% 問題文で与えられた条件を考慮すると、正しい計算方法と正確な組み合わせを考えることが重要です。\\
% 誤答のプログラムでは、組み合わせを取得しているようですが、その後の計算が誤っています。ここでいくつかのヒントを提供します:\\
% 1 積を計算するときに演算子(*)を使う代わりに剰余演算子(\%)を適切に使ってください。誤答のプログラムでは、積を計算する際に*を用いていますが、問題では積をPで割った余りがQであるかを求める必要があります。\\
% 2 正しい組み合わせを選ぶために、ループのインデックスを使用して値を取得していますが、これらの値をそのまま使用するのではなく、リストAから対応する要素を取り出す必要があります。\\
% 3 組み合わせを取得する際には、それぞれの数が異なることを確認する必要があります。誤答のプログラムでは、各インデックスが異なることを確認するコードがありません。
% \end{screen}
% このヒントを確認すると、正答について述べず、2のヒントで的確な指摘を行えています。
% しかし、誤答のプログラムは演算子を正しく使えてるが、1のヒントで間違っているような提案を行っている。
% また、3のヒントは確認を行う必要がない操作を追加させるよう指示している。
% したがって、解答者を混乱させる可能性がある。
% 上記より、正しい指摘と間違った指摘が混合し解答者を混乱させるようなヒントを提案している。

% \section{今後うまくいきそうな例2}
% % プログラムに対する指摘自体はいいものになっているけど、
% % プログラムのロジック的な間違いに触れられていない例
% % 8->9のn=0空白段階がこの例。
% % 決定!!!
% % よかった例3.1と同様の問題とプログラム
% \subsection{ヒント提案}
% この例では3.1と同様の問題文,正答,誤答と入力パターン2で入力した。x
% Chat GPTが提案したヒントは以下である。
% \begin{screen}
% \textbf{出力}\\
% プログラム全体を示さないようにしますが、この問題の正答を考える上でいくつかの修正点があります。与えられたプログラムには、操作のループ内で使われている変数iが重複しており、内側のループと外側のループで同じ変数名が使用されています。これにより内側のiが外側のiによって上書きされ、意図しない挙動を引き起こす可能性があります。
% また、Pythonのコードでは変数iはスコープ(有効範囲)がループ内に限定されるため、外側のループで使われる変数iと内側のループで使われる変数iは異なる変数として扱われます。そのため、内側のループでiを使う際には別の変数名を使用することが望ましいです。
% これらの問題点を修正するために、内側のループで使われる変数名を外側のループで使われている変数とは異なる名前に変更すると、プログラムの動作が改善されるでしょう。
% \end{screen}
% このヒントを確認すると、for文の変数のスコープに対するプログラムを記述するうえでの
% 注意点を述べた素晴らしいヒントである。
% しかし、誤答を正答に導くという観点に関しては空白文字が出力されてしまうこと
% について指摘を行えていないところが懸念点である。

% \section{今後うまくいきそうなヒント3}%答えバラしちゃう(悪いというよりも今後のヒント)}
% % 答えバラしちゃう系
% % 行列総和TLE優秀の答え言い過ぎはいい例かも
% % 決定!!!
% この例で取り扱う問題は参考文献[16]に記載されている。\\
% 問題文,正答,誤答は以下である。
% \subsection{問題文}
% \begin{screen}
% \begin{math}
% H行 W列のマス目があります。上から i (1\textless=i\textless=H) 行目、左から j (1\textless=j\textless=W)列目にあるマス(i,j)には、
% 整数Aijが書かれています。 すべてのマス(i,j) (1\textless=i\textless=H, 1\textless=j\textless=W) について、以下の値を求めてください。\\
% マス(i,j)と同じ行または同じ列にあるマス(自分自身を含む)に書かれている整数をすべて合計した値\\
% 制約\\
% ・2\textless=H,W\textless=2000\\
% ・1\textless=Aij\textless=99\\
% ・入力は全て整数
% \end{math}
% \end{screen}

% \subsection{正答}
% \begin{lstlisting}
% H, W = map(int, input().split())
% m = [list(map(int, input().split())) for _ in range(H)]
    
% ans_m = [[] for _ in range(H)]
% yoko = list(map(sum, m))
% tate = list(map(sum, zip(*m)))
% for i in range(H):
%     for j in range(W):
%         S = yoko[i] + tate[j] - m[i][j]
%         ans_m[i].append(S)
% for k in range(H):
%     print(' '.join(map(str, ans_m[k])))
% \end{lstlisting}
% このプログラムはpanicaの提出結果を利用した。[17]
% 5,6行目で行,列各々の累積和を利用することで、9行目で効率的に計算を行っている\\

% \subsection{誤答}
% \begin{lstlisting}
% H,W=map(int,input().split())
% A=[list() for _ in range(H+1)]
% for i in range(1,H+1):
%     A[i]=list(map(int,input().split()))
% B=[[0]*W for _ in range(H+1)]
% for i in range(1,H+1):
%     for j in range(W):
%         B[i][j]+=sum(A[i])
%         for h in range(1,H+1):
%             B[i][j]+=A[h][j]
%         B[i][j]-=A[i][j]
% for i in range(1,H+1):
%     print(B[i])
% \end{lstlisting}

% \begin{math}
% このプログラムはvivantの提出結果を利用した。[18]
% 6\sim11行目で各マスにおける総和を計算しているが、
% これでは計算時間が掛かり過ぎてしまい、実行時間に間に合わずTLEになる。
% \end{math}

% \subsection{ヒント提案}
% 問題文,正答,誤答を入力パターン1で入力を行った結果,
% Chat GPTが提案したヒントは以下である。
% \begin{screen}
% \textbf{出力}\\
% 誤答のプログラムでは、2つの2重ループを使用して計算しており、計算量が増加しているため、効率的な方法を提案します。\\
% 基本的な考え方は、行ごとと列ごとの合計を事前に計算しておき、それを利用して各マス(i, j)の計算を行うことです。
% 以下は修正のステップバイステップのヒントです。\\
% 1 各行および列の合計を計算します。行ごとの合計を保存するリスト(例:yoko)と
% 列ごとの合計を保存するリスト(例:tate)を作成します。これは、正答のプログラムでも行われています。\\
% 2 各マス(i, j)について、行の合計(yoko[i])と列の合計(tate[j])を足し、
% マス(i, j)の値(A[i][j])を引きます。これにより、条件を満たす合計が得られます。\\
% 3 修正後の計算を新しいリストに保存し、それを出力形式に合わせて出力します。\\
% 修正後のプログラムは、正答のプログラムと同様に正しい結果を生成しますが、
% 計算量が削減され、効率的な方法で問題を解決します。\\
% \end{screen}
% このヒントを確認すると、最初の一文で計算量が増加しているといういい指摘を行っている。
% しかし、修正のヒントが正答を教えている。
% 故にこのヒントは誤答の問題点を解説するところまではいいが答えを教えてしまうものになっている。\\

% \section{悪い例1}
% % 的外れのテキトウなヒント系
% % イルミネーション場合分けの優秀がこれっぽい
% % 決定!!!
% この例で取り扱う問題は参考文献[19]に記載されている。\\
% 問題文,正答,誤答は以下である。
% \subsection{問題文}
% \begin{screen}
% \begin{math}
% \textbf{問題文}\\
% E869120 くんは、冬に公開するイルミネーションを作成することを計画しています。
% E869120 くんが計画しているイルミネーションは、縦H×横WのHW個のLEDで構成されます。
% イルミネーションの各LEDは、点灯・消灯の状態を任意に切り替えることができます。
% このイルミネーションは、以下の条件を満たすとき 不適切である といいます。
% ・イルミネーション全体に完全に含まれる 縦 2 × 横 2の、4つの LED を含む領域であって、点灯している LED が領域内に 2 つ以上あるものが存在する。
% 適切な(不適切な状態ではない)イルミネーションの点灯パターンのうち、点灯している LED の個数としてありうる最大値を求めてください。\\
% 制約
% ・1\textless=H,W\textless=100	\\
% ・入力はすべて整数\\
% \end{math}
% \end{screen}

% \subsection{正答}
% \begin{lstlisting}
% H, W = map(int, input().split())

% if H == 1 or W == 1:
%     print(H * W)
% else:
%     print(((W + 1) // 2) * ((H + 1) // 2))
% \end{lstlisting}
% \begin{math}
% このプログラムはotafukuの提出結果を利用した。[20]
% 行か列が1の場合はすべてのマスにLEDが置けること3,4行目で記述している。\\
% その他の場合、領域内の左上にLED置き、1マスずつ間隔を開けて配置できるため、
% 6行目の計算方法により正答を得られる。
% \end{math}

% \subsection{誤答}
% \begin{lstlisting}
% H, W = map(int, input().split())

% print(((H+1)//2) * ((W+1)//2))
% \end{lstlisting}
% このプログラムはSHIN10101の提出結果を利用した。[21]
% 行か列が1における場合分けが行えていないため、誤っている。


% \subsection{ヒント提案}
% 問題文,正答,誤答を入力パターン1で入力を行った結果,
% Chat GPTが提案したヒントは以下である。
% \begin{screen}
% \textbf{出力}\\
% 問題では、イルミネーション全体に含まれる2×2の領域で、点灯しているLEDが2つ以上ある場合を不適切としています。そのため、不適切な状態を回避しつつ、点灯するLEDの個数を最大化する方法を見つける必要があります。\\
% これを達成するためのアプローチの1つは、イルミネーションをグループに分割することです。特定のパターンで2×2の領域に2つ以上の点灯LEDが含まれないようにイルミネーションを構築することがポイントです。各グループ内の点灯LEDの数を最大化しつつ、2×2の領域での条件を満たすようなグループ配置を考えてみてください。
% \end{screen}
% このヒントを確認すると、行または列が1の時について言及していないことから的外れなヒントである。\\


% \section{悪い例2}
% % あってることを間違ってるっていう嘘ついちゃう系
% % ケーキ立方体のfloatがこれっぽい(問題+誤答)
% % 決定!!!
% この例では4.4と同様の問題を利用する。
% \subsection{誤答}
% \begin{lstlisting}
% H,W = map(int,input().split())

% if H == 1 or W == 1:
%     print(H*W)
% else:
%     print((H+1)//2 * (W+1)//2)
% \end{lstlisting}

% このプログラムはtks\_の提出結果を利用した。[22]
% 6行目の最大数を求めるところで適切に括弧をつけられていない。
% これにより、左結合となり(H+1)//2*(W+1)を行った後//2を行ってしまうため正しい計算を行えていない。

% \subsection{ヒント提案}
% 問題文+誤答を入力パターン3で入力した。
% Chat GPTが提案したヒントは以下である。
% \begin{screen}
% \textbf{出力}\\
% このプログラムでは、条件を考慮せずに単純に計算していますね。誤答のプログラムを修正するためのヒントを提供しましょう。\\
% 条件では、縦2×横2の領域内に点灯しているLEDが2つ以上ある場合、その状態は不適切とされています。そのため、それを考慮する必要があります。\\
% ここで考えるべきポイントは、どのような点灯パターンが条件を満たさないかです。条件を満たさないパターンを除外しつつ、最大値を求めるアプローチを考えてみてください。特に、縦2×横2の領域内の点灯パターンに着目することが重要です。
% \end{screen}
% このヒントを確認すると、括弧を付けてないことによる計算順序間違いについて指摘していない。
% それに加え、提案したヒントが特に意味のない内容となっている。\\



% \section{悪い例3}
% % あってることを間違ってるっていう嘘ついちゃう系
% % ケーキ立方体のfloatがこれっぽい(優秀,段階両方)
% % 決定!!!
% この例で取り扱う問題は参考文献[23]に記載されている。\\
% 問題文,正答,誤答は以下である。

% \subsection{問題文}
% \begin{screen}
% \begin{math}
% \textbf{問題文}\\
% 幅 A、奥行き B、高さ C の直方体の形をしたケーキがあります。\\
% あなたはケーキに対して、次の操作を行うことができます。\\
% ・ある面に平行な方向に切断する\\
% ・ただし、ケーキは動かしてはならない(複数のケーキに分割されている場合、これらを変形したり、別々に切ることはできない)\\
% 最小何回の操作で、全てのピースを立方体にすることができますか?\\
% 制約\\
% ・1\textless=A,B,C\textless=10^{18}\\
% ・入力はすべて整数
% \end{math}
% \end{screen}

% \subsection{正答}
% \begin{lstlisting}
% import math
    
% a, b, c = map(int, input().split())   
% gcd = math.gcd(a, b, c)   
% print((a//gcd-1)+(b//gcd-1)+(c//gcd-1))
% \end{lstlisting}
% \begin{math}
% このプログラムは39toの提出結果を利用した。[24]
% この問題において、立方体の1辺の長さはA,B,Cの最大公約数となる。
% この操作は3行目で行われている。
% そして、4行目のように各辺を最大公約数で割ることにより
% 切る回数を求めることで正解を得られる。
% \end{math}

% \subsection{誤答}
% \begin{lstlisting}
% import math
% a, b, c = map(int, input().split())
% tmp = math.gcd(a, math.gcd(b, c))
% print(int((a/tmp - 1)+(b/tmp - 1)+(c/tmp - 1)))
% \end{lstlisting}
% \begin{math}
% このプログラムはfumi8の提出結果を利用した。[25]
% 4行目の切る回数を求めるところで/演算子を利用している。
% この演算子では浮動小数点を扱うため、値が大きくなるほど誤差が生じてしまうという誤りである。
% \end{math}

% \subsection{ヒント提案}
% 問題文,正答,誤答を入力パターン1で入力を行った結果,
% Chat GPTが提案したヒントは以下である。
% \begin{screen}
% \textbf{出力}\\
% 誤答プログラムを正しいプログラムに修正するヒントを提供します。\\
    
% 誤答プログラムでは、最初にaとbをGCDで結びつけ、その結果とcのGCDを計算しようとしています。
% しかし、これは問題文に従った操作とは異なります。
% 正しいアプローチは、3つの寸法 A, B, C の GCD を計算することです。\\
% \end{screen}
% このヒントを確認すると、浮動小数点に関する指摘が一切ない。
% 加えて、最大公約数の求め方が間違っているという嘘の指摘を行っている
% のがよくない。

% % 〇と×で対照的な問題を探すのもいい感じになりそう。
% % ・nC5添え字の優秀と段階はいい例かも
% % ・8->10->9と隣接は優秀と段階で天と地の差がありそうでいい例にできるかも。

